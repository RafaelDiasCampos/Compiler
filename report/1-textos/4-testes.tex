\chapter{Metodologia dos testes}
\label{cap:testes}

Para validar a implementação do compilador, foram realizados testes com alguns códigos.
Inicialmente, foi feito o teste com um código que contém erros de natureza Léxica, Sintática e/ou Semântica.
Em seguida, esse código teve seus erros corrigidos e foi realizado novamente o teste.
Nas próximas seções estão descritos os códigos que foram utilizados:

\section{Teste 01}
\label{subsec:teste01}

\subsection{Código Original}
\verbatiminput{../tests/1/original.txt}

\subsection{Código Corrigido}
\verbatiminput{../tests/1/fixed.txt}

\section{Teste 02}
\label{subsec:teste02}

\subsection{Código Original}
\verbatiminput{../tests//original.txt}

\subsection{Código com Comentário Corrigido}
\verbatiminput{../tests/2/fixedComment.txt}

\subsection{Código Corrigido}
\verbatiminput{../tests/2/fixed.txt}

\section{Teste 03}
\label{subsec:teste03}

\subsection{Código Original}
\verbatiminput{../tests/3/original.txt}

\subsection{Código Corrigido}
\verbatiminput{../tests/3/fixed.txt}

\section{Teste 04}
\label{subsec:teste04}

\subsection{Código Original}
\verbatiminput{../tests/4/original.txt}

\subsection{Código com String Corrigida}
\verbatiminput{../tests/4/fixedString.txt}

\subsection{Código Corrigido}
\verbatiminput{../tests/4/fixed.txt}

\section{Teste 05}
\label{subsec:teste05}

\subsection{Código Original}
\verbatiminput{../tests/5/original.txt}

\subsection{Código com String Corrigida}
\verbatiminput{../tests/5/fixedString.txt}

\subsection{Código Corrigido}
\verbatiminput{../tests/5/fixed.txt}

\section{Teste 06}
\label{subsec:teste06}

\subsection{Código Original}
\verbatiminput{../tests/6/original.txt}

\subsection{Código Corrigido}
\verbatiminput{../tests/6/fixed.txt}

\section{Teste 07}
\label{subsec:teste07}

\subsection{Código Original}
\verbatiminput{../tests/7/original.txt}

\subsection{Código Corrigido}
\verbatiminput{../tests/7/fixed.txt}