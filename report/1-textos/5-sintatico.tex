\chapter{Analisador Sintático}
\label{cap:sintatico}

\section{Alteração na Gramática}
\label{sec:sintaticoAlteracaoGramatica}
A gramática original apresenta recursão a esquerda, o que impossibilita a construção de um parser recursivo descendente que a reconheça.
Por isso, foi feita a seguinte modificação na gramática para eliminar essa recursão:

\begin{grammar}
    
    <program> ::= `routine' body
    
    <body> ::= [<decl-list>] `begin' <stmt-list> `end'
    
    <decl-list> ::= `declare' <decl> `;' {<decl> `;'}
    
    <decl> ::= <type> <ident-list>
    
    <ident-list> ::= <identifier> {`;' <identifier>}
    
    <type> ::= `int` | `float' | `char'
    
    <stmt-list> ::= <stmt> {`;' <stmt>}
    
    <stmt> ::= <assign-stmt> | <if-stmt> | <while-stmt> | <repeat-stmt> | <read-stmt> | <write-stmt>
    
    <assign-stmt> ::= <identifier> `:=' <simple-expr-a>
    
    <if-stmt> ::= `if' <condition> `then' <stmt-list> `end'
                  \alt `if' <condition> `then' <stmt-list> `else' <stmt-list> `end'
                  
    <condition> : <expression>
    
    <repeat-stmt> ::= `repeat' <stmt-list> <stmt-suffix>
    
    <stmt-suffix> ::= `until' <condition>
    
    <while-stmt> ::= <stmt-prefix> <stmt-list> `end'
    
    <stmt-prefix> ::= `while' <condition> `do'
    
    <read-stmt> ::= `read' `(' <identifier> `)'
    
    <write-stmt> ::= `write` `(` <writable> `)'
    
    <writable> ::= <simple-expr-a> | <literal>
    
    <expression> ::= <simple-expr-a> | <simple-expr-a> <relop> <simple-expr-a>
    
    <simple-expr-a> ::= <term-a> <simple-expr>

    <simple-expr> ::= <addop> <term-a> <simple-expr> | lambda
    
    <term-a> ::= <factor-a> <term>

    <term> ::= <mulop> <factor-a> <term> | lambda
    
    <factor-a> ::= <factor> | `not' <factor> | `-' <factor>
    
    <factor> ::= <identifier> | <constant> | `(' <expression> `)'
    
    <relop> ::= `=' | `>' | `>=' | `<' | `<=' | `<>'
    
    <addop> ::= `+' | `-' | `or'
    
    <mulop> ::= `*' | `/' | `and'
    
    <constant> ::= <integer_const> | <float_const> | <char_const>
    
    \end{grammar}

\section{Calculo de First e Follow}
\label{sec:sintaticoFirstFollow}

Para ajudar na implementação do compilador, foi realizad o cálculo do First e Follow para cada não-terminal da gramática:

    \begin{itemize}
        \item \textbf{First}:
        \begin{itemize}
            \item \textit{<program>}: `routine'
            \item \textit{<body>}: `begin' | `declare'
            \item \textit{<decl-list>}: `declare'
            \item \textit{<decl>}: `int' | `float' | `char'
            \item \textit{<ident-list>}: <identifier>
            \item \textit{<type>}: `int' | `float' | `char'
            \item \textit{<stmt-list>}: <identifier> | `if' | `while' | `repeat' | `read' | `write'
            \item \textit{<stmt>}: <identifier> | `if' | `while' | `repeat' | `read' | `write'
            \item \textit{<assign-stmt>}: <identifier>
            \item \textit{<if-stmt>}: `if'
            \item \textit{<condition>}: <identifier> | `(' | `not' | `-' | <integer_const> | <float_const> | <char_const>
            \item \textit{<repeat-stmt>}: `repeat'
            \item \textit{<stmt-suffix>}: `until'
            \item \textit{<while-stmt>}: `while'
            \item \textit{<stmt-prefix>}: `while'
            \item \textit{<read-stmt>}: `read'
            \item \textit{<write-stmt>}: `write'
            \item \textit{<writable>}: <identifier> | `(' | `not' | `-' | <literal> | <integer_const> | <float_const> | <char_const>
            \item \textit{<expression>}: <identifier> | `(' | `not' | `-' | <integer_const> | <float_const> | <char_const>
            \item \textit{<simple-expr-a>}: <identifier> | `(' | `not' | `-' | <integer_const> | <float_const> | <char_const>
            \item \textit{<simple-expr>}: `-' | `+' | `or' | lambda
            \item \textit{<term-a>}: <identifier> | `(' | `not' | `-' | <integer_const> | <float_const> | <char_const>
            \item \textit{<term>}: `*' | `/' | `and' | lambda
            \item \textit{<factor-a>}: <identifier> | `(' | `not' | `-' | <integer_const> | <float_const> | <char_const>
            \item \textit{<factor>}: <identifier> | `(' | <integer_const> | <float_const> | <char_const>
            \item \textit{<relop>}: `=' | `>' | `>=' | `<' | `<=' | `<>'
            \item \textit{<addop>}: `+' | `-' | `or'
            \item \textit{<mulop>}: `*' | `/' | `and'
            \item \textit{<constant>}: <integer_const> | <float_const> | <char_const>
        \end{itemize}
        \item \textbf{Follow}:
        \begin{itemize}
            \item \textit{<program>}: \$
            \item \textit{<body>}: \$
            \item \textit{<decl-list>}: `begin'
            \item \textit{<decl>}: `;'
            \item \textit{<ident-list>}: `;'
            \item \textit{<type>}: <identifier> 
            \item \textit{<stmt-list>}: `end' | `else' | `until'
            \item \textit{<stmt>}: `end' | `else' | `until'
            \item \textit{<assign-stmt>}: `end' | `else' | `until'
            \item \textit{<if-stmt>}: `end' | `else' | `until'
            \item \textit{<condition>}: `end' | `else' | `until' | `then' | `do'
            \item \textit{<repeat-stmt>}: `end' | `else' | `until'
            \item \textit{<stmt-suffix>}: `end' | `else' | `until'
            \item \textit{<while-stmt>}: `end' | `else' | `until'
            \item \textit{<stmt-prefix>}: <identifier> | `if' | `repeat' | `while' | `read' | `write'
            \item \textit{<read-stmt>}: `end' | `else' | `until' 
            \item \textit{<write-stmt>}: `end' | `else' | `until' 
            \item \textit{<writable>}: `)'
            \item \textit{<expression>}: `end' | `then' | `else' | `until' | `do' | `)'
            \item \textit{<simple-expr-a>}: `end' | `then' | `else' | `until' | `do' | `)' | `=' | `>' | `>=' | `<' | `<=' | `<>'
            \item \textit{<simple-expr>}: `end' | `then' | `else' | `until' | `do' | `)' | `=' | `>' | `>=' | `<' | `<=' | `<>'
            \item \textit{<term-a>}: `end' | `then' | `else' | `until' | `do' | `)' | `-' | `+' | `or' | `=' | `>' | `>=' | `<' | `<=' | `<>'
            \item \textit{<term>}: `end' | `then' | `else' | `until' | `do' | `)' | `-' | `+' | `or' | `=' | `>' | `>=' | `<' | `<=' | `<>'
            \item \textit{<factor-a>}: `end' | `then' | `else' | `until' | `do' | `)' | `-' | `+' | `or' | `=' | `>' | `>=' | `<' | `<=' | `<>' | `*' | `/' | `and'
            \item \textit{<factor>}: `end' | `then' | `else' | `until' | `do' | `)' | `-' | `+' | `or' | `=' | `>' | `>=' | `<' | `<=' | `<>' | `*' | `/' | `and'
            \item \textit{<relop>}: <identifier> | `(' | `not' | `-' | <integer_const> | <float_const> | <char_const>
            \item \textit{<addop>}: <identifier> | `(' | `not' | `-' | <integer_const> | <float_const> | <char_const>
            \item \textit{<mulop>}: <identifier> | `(' | `not' | `-' | <integer_const> | <float_const> | <char_const>
            \item \textit{<constant>}: `end' | `then' | `else' | `until' | `do' | `)' | `-' | `+' | `or' | `=' | `>' | `>=' | `<' | `<=' | `<>' | `*' | `/' | `and'
        \end{itemize}
    \end{itemize}
    
\section{Estrutura do Programa}
\label{sec:sintaticoEstrutura}
Foi feita implementação de um Parser Recursivo Descendente para reconhecer a gramática da linguagem tilizando o algoritmo de parsing LL(1) .

Cada regra da gramática foi representada por uma classe que herda de \textbf{Construct}.
Essas regras recebem como parâmetros objetos do tipo \textbf{Token} ou do tipo \textbf{Construct} que representam a derivação escolhida para essa regra. 

\section{Recuperação de Erros}
Para recuperar os erros de sintaxe, foi utilizado o método baseado na análise do Follow de cada regra da gramática.
Ao encontrar um erro na construção de uma regra, o programa "come"  novos Tokens até encontrar um que pertence ao Follow da regra que estava sendo construída.
Dessa forma, é garantido que pelo menos o início da construção da regra seguinte será possível, o que permite um reconhecimento maior de erros.


\section{Testes de validação}
\label{sec:sintaticoTestes}

\subsection{Teste 01}
\label{subsec:sintaticoTeste01}

\subsubsection{Código Original}
\verbatiminput{../tests/1/original.txt}

Saida do Compilador:

\verbatiminput{3-testes/sintatico/1/original.txt}

\subsubsection{Código Original Corrigido}
\verbatiminput{../tests/1/fixedSyntatic.txt}

Saida do Compilador:

\verbatiminput{3-testes/sintatico/1/fixedSyntatic.txt}

\subsection{Teste 02}
\label{subsec:sintaticoTeste02}

\subsubsection{Código Original}
\verbatiminput{../tests/2/fixedLexic.txt}

Saida do Compilador:


\verbatiminput{3-testes/sintatico/2/fixedLexic.txt}

\subsubsection{Código Corrigido}
\verbatiminput{../tests/2/fixedSyntatic.txt}

Saida do Compilador:


\verbatiminput{3-testes/sintatico/2/fixedSyntatic.txt}

\subsection{Teste 03}
\label{subsec:sintaticoTeste03}

\subsubsection{Código Original}
\verbatiminput{../tests/3/original.txt}

Saida do Compilador:


\verbatiminput{3-testes/sintatico/3/original.txt}
\subsection{Teste 04}

\subsubsection{Código Corrigido}
\verbatiminput{../tests/3/fixedSyntatic.txt}

Saida do Compilador:


\verbatiminput{3-testes/sintatico/3/fixedSyntatic.txt}

\subsection{Teste 04}
\label{subsec:sintaticoTeste04}

\subsubsection{Código Original}
\verbatiminput{../tests/4/fixedLexic.txt}

Saida do Compilador:


\verbatiminput{3-testes/sintatico/4/fixedLexic.txt}

\subsubsection{Código Corrigido}
\verbatiminput{../tests/4/fixedSyntatic.txt}

Saida do Compilador:


\verbatiminput{3-testes/sintatico/4/fixedSyntatic.txt}

\subsection{Teste 05}
\label{subsec:sintaticoTeste05}

\subsubsection{Código Original}
\verbatiminput{../tests/5/fixedLexic.txt}

Saida do Compilador:


\verbatiminput{3-testes/sintatico/5/fixedLexic.txt}

\subsubsection{Código com Token Routine}
\verbatiminput{../tests/5/fixedRoutine.txt}

Saida do Compilador:


\verbatiminput{3-testes/sintatico/5/fixedRoutine.txt}


\subsubsection{Código Corrigido}
\verbatiminput{../tests/5/fixedSyntatic.txt}

Saida do Compilador:


\verbatiminput{3-testes/sintatico/5/fixedSyntatic.txt}

\subsection{Teste 06}
\label{subsec:sintaticoTeste06}

\subsubsection{Código Original}
\verbatiminput{../tests/6/fixedLexic.txt}

Saida do Compilador:


\verbatiminput{3-testes/sintatico/6/fixedLexic.txt}

\subsubsection{Código Corrigido}
\verbatiminput{../tests/6/fixedSyntatic.txt}

Saida do Compilador:


\verbatiminput{3-testes/sintatico/6/fixedSyntatic.txt}

\subsection{Teste 07}
\label{subsec:sintaticoTeste07}

\subsubsection{Código Original}
\verbatiminput{../tests/7/fixedLexic.txt}

Saida do Compilador:


\verbatiminput{3-testes/sintatico/7/fixedLexic.txt}

\subsubsection{Código Corrigido}
\verbatiminput{../tests/7/fixedSyntatic.txt}

Saida do Compilador:


\verbatiminput{3-testes/sintatico/7/fixedSyntatic.txt}