\chapter{Descrição da Linguagem}
\label{cap:descricaoLinguagem}

\section{Gramática}
\label{sec:gramatica}

A linguagem foi definida pela seguinte Gramática Livre de Contexto (em notação BNF):

\begin{grammar}

<statement> ::= <ident> `=' <expr>

<program> ::= <a> `=' `routine' body

<body> ::= [<decl-list>] `begin' <stmt-list> `end'

<decl-list> ::= `declare' <decl> `;' {<decl> `;'}

<decl> ::= `declare' <decl> `;' {<decl> `;'}

<ident-list> ::= <identifier> {`;' <identifier>}

<type> ::= `int` | `float' | `char'

<stmt-list> ::= <stmt> {`;' <stmt>}

<stmt> ::= <assign-stmt> | <if-stmt> | <while-stmt> | <repeat-stmt> | <read-stmt> | <write-stmt>

<assign-stmt> ::= <identifier> `:=' <simple-expr>

<if-stmt> ::= `if' <condition> `then' <stmt-list> `end'
              \alt `if' <condition> `then' <stmt-list> `else' <stmt-list> `end'
              
<condition> : <expression>

<repeat-stmt> ::= `repeat' <stmt-list> <stmt-suffix>

<stmt-suffix> ::= `until' <condition>

<while-stmt> ::= <stmt-prefix> <stmt-list> `end'

<stmt-prefix> ::= `while' <condition> `do'

<read-stmt> ::= `read' `(' <identifier> `)'

<write-stmt> ::= `write` `(` <writable> `)'

<writable> ::= <simple-expr> | <literal>

<expression> ::= <simple-expr> | <simple-expr> <relop> <simple-expr>

<simple-expr> ::= <term> | <simple-expr> <addop> <term>

<term> ::= <factor-a> | <term> <mulop> <factor-a>

<fator-a> ::= <factor> | `not' <factor> | `-' <factor>

<factor> ::= <identifier> | <constant> | `(' <expression> `)'

<relop> ::= `=' | `>' | `>=' | `<' | `<=' | `<>'

<addop> ::= `+' | `-' | `or'

<mulop> ::= `*' | `/' | `and'

<constant> ::= <integer_const> | <float_const> | <char_const>

\end{grammar}

\section{Tokens}
\label{sec:tokens}

Os tokens foram definidos a partir das seguintes expressões regulares:

\begin{grammar}

<integer_const> ::= <digit>+

<float_const> ::= <digit>+ `.' <digit>+

<char_const> ::= `\'' <carac> `\''

<literal> ::= `\"' <caractere>* `\"'

<identifier> ::= <letter> (<letter> | <digit>)*

<letter> ::= [A-za-z]

<digit> ::= [0-9]

<carac> ::= .*

\end{grammar}

\section{Outras características}
\label{sec:caracteristicas}

A linguagem apresenta algumas outras características, descritas a seguir:

\begin{itemize}
\item Palavras-chave são reservadas
\item Variáveis devem ser declaradas antes de seu uso
\item Comentários se iniciam por '\%' e terminal com '\%'
\item Valores do tipo inteiro podem ser atribuídos a variáveis do tipo float, mas o inverso não é permitido
\item O resultado de uma divisão é sempre um float
\item A linguagem é case-sensitive
\end{itemize}